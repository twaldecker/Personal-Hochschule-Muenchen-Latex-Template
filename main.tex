%% Standard-Vorspann
\documentclass[a4paper,11pt,oneside]{article}
\usepackage{ngerman}                   
\usepackage[utf8]{inputenc}
\usepackage[T1]{fontenc}
\usepackage[fleqn]{amsmath}

%% Auf Schriftart Palatino umschalten
\usepackage{mathpazo}
\usepackage[scaled=.95]{helvet}
\usepackage{courier}

%% fast immer benoetigte Pakete
\usepackage{epsfig}
\usepackage{amssymb}
\usepackage{alltt}
\usepackage{latexsym}
\usepackage{makeidx}
\usepackage{textcomp}
\usepackage{rotating}
\usepackage{color}
\usepackage{caption}
\usepackage{verbatim}
\usepackage{fancyhdr}

\usepackage[
  plainpages=false,
  unicode=true,          % non-Latin characters in Acrobat?s bookmarks
  pdftoolbar=true,        % show Acrobat?s toolbar?
  pdfmenubar=true,        % show Acrobat?s menu?
  pdffitwindow=true,     % window fit to page when opened
  pdfstartview={FitV},    % fits the width of the page to the window
  pdfnewwindow=true,      % links in new window
  colorlinks=true,       % false: boxed links; true: colored links
  linkcolor=red,          % color of internal links
  citecolor=green,        % color of links to bibliography
  filecolor=magenta,      % color of file links
  urlcolor=cyan,          % color of external links
  hyperfootnotes=false,
  bookmarks,
]{hyperref}
 							
%% spezielles Zeug
\usepackage{Abschlussarbeit}

%% Index erzeugen
\makeindex

%% jetzt geht's los
\begin{document}
\raggedbottom


% Vorspann der Arbeit
%
\begin{titlepage}
\begin{flushright}
\includegraphics[width=70mm]{img/hm.jpg}%
\end{flushright}

\vspace*{20mm}
\begin{center}
{\Large Hochschule München}\\
{\large Fakultät für Mathematik und Informatik}\\

\vspace*{15mm}
{\huge Seminararbeit }\\

\vspace*{10mm}
{\huge \bfseries{ Hier der Titel }} \\
\vspace*{15mm} 
\end{center}

\vspace*{30mm}

\begin{tabular}{lll}
\textbf{\large {Autor:}} & & \large {Thomas Waldecker}\\
& & \\

\textbf{\large {Abgabe:}} & & \large {10.01.2012}\\
& & \\

\textbf{\large {betreut von:}} & & \large {Prof. Dr. No}\\
& & \\
\end{tabular}

\end{titlepage}


\thispagestyle{empty}
\clearpage

\begin{center}
{\Large \bfseries{ Kurzfassung }}\\
\end{center}

Hier die Kurzfassung in Deutsch einfügen
xxxxxxxxxxxxxxxxxxxxxxxxxxxx


\vspace*{10mm}
\begin{center}
{\Large \bfseries{ Abstract }}\\
\end{center}

Hier die Kurzfassung in Englisch einfügen
xxxxxxxxxxxxxxxxxxxxxxxxxxxx

\leereseite
%% Inhaltsverzeichnis
\tableofcontents
\leereseite
  
\section{Einleitung}

Im Rahmen des Hauptseminars für Embedded Computing, einem Masterstudiengang im Fach Informatik der Hochschule München, wurde von fünf Studierenden ein universelles Kommunikationsframework für Smart Objects entwickelt.
\cite{Akella2010}


\subsection{Anwendungsbereich Parkleitsystem}

Bestehende Systeme zeigen die Anzahl der freien Plätze auf Schilder an und Fahrer versuchen dann den entsprechenden Parkplatz oder das Parkhaus zu finden. Möchte der Fahrer in einer fremden Stadt mit einem Navigationsgerät einen Parkplatz finden, muss er nachdem er das Parkleitsystem gesehen hat den entsprechenden Parkplatz erst in sein Gerät eingeben [3].

%\include{kap1}


%% Beim Anhang gibt es eine Musterdatei, die 
%% ergänzt werden kann
\appendix

\glsaddall % alle definierten abk�rzungen zeigen
\printglossary[numberedsection, title=Glossar] %Glossar, section mit nummer.

\bibliography{verzeichnis}
\bibliographystyle{alpha}


%% Stichwortverzeichnis ausgeben.
%% falls nicht notwendig, die folgende Zeile loeschen
\printindex

\end{document}
